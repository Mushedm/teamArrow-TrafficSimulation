\chapter{Project Description}

\section{Aims of the project and structure}

\vspace{1em}
\noindent The aim of this project is to develop a data visualization tool to display how vehicles interact with each other and a particular environment as well as simulate a variety of traffic paradigms. This tool will be primarily developed in JAVA and will be divided into multiple layers with each layer having a specific objective. The overall purpose of the layers is to divide complex parts of the tool into simpler parts that will allow the team to program more efficiently. The team has decided that the tool will initially consist of the following five layers:


\begin{enumerate}
  \item \textbf{Vehicle Layer} -  this layer will consist of the various types of vehicles that will be given simple commands to follow and situations to react to.
  \item \textbf{Map Layer} - this layer will be the part of the tool that displays the roads, roundabouts and other routes that a vehicle can interact with. This layer will not feature any vehicles
  \item \textbf{Interaction Layer} - this layer will combine the vehicle and map layers. It will allow the team to test if a vehicle can successfully follow a certain route or not.
  \item \textbf{Event Layer} - this layer will add events such as traffic jams, road emergencies, etc to the tool. 
  \item \textbf{Simulation Layer} -  this final layer will involve the users selecting a range of maps and events to simulate a certain paradigm that they wish to model. 
\end{enumerate}

\vspace{1em}
\section{Strategy for achieving project aims}
\noindent
Each layer has a specific purpose towards the tool and is necessary that a layer is completed before advancing to the next layer. This approach is being taken to ensure that every layer has a strong foundation and has the absolute minimum amount of errors possible. The vehicle layer and map layer form the basic part of the tool therefore the team will focus on developing these two aspects first. If these two layers are not functioning properly then the interaction layer will not work as desired and will cause other layers problems in later stages of development. Through this we believe that it will be easier to catch errors or identify areas that need to further improved.

\vspace{1em}
\noindent
The event and simulation layers will serve as the advanced part of the tool and focus on meeting the main objectives of the project. These two layers will require more focus than the others and can only be developed when the interaction layer functions properly without errors. As the event layer will introduce the concepts of road emergencies and restrictions it will help the team make a more realistic traffic simulation engine. The simulation will be the last layer to be developed and will allow the user to vies various traffic scenarios and how they could play out.  